\documentclass[../eva1_scion.tex]{subfiles}
\begin{document}
    \section{Discussion}\label{ch:discussion}

    After we have looked at the componants and processes which constitute the SCION architecture, in this section we discuss how SCION manages to provide the quality metrics defined in the introduction (see section \ref{sec:introduction}) and look at further emergent properties of the architecture.

    \subsection{Availablilty}
    SCION incorperates multiple direct designe descissions which work towards improving availablity of the network, as well as other properties of the designe which inderectly contribute towards that goal.

    In SCIONs down-path registration mechanism makes SCION inherently multipathed, since each AS may register multiple down-paths with the core path servers. This controll over down paths also allows an AS to select down paths for their reliablity and may select paths in a way to avoid unreliable ASes. Further in the endhost may take relyability into account during path construction as well. Finally each packet carries its routing information in the packet header and the source may select a path an a per packet basis, making use of multipath nature of SCION.

    The high path freshness by guaranteed by SCIONs regular beaconing process further works in favour of availability, since valid and working paths are propagated through an ISD every few seconds. Physical route failures are detected quickly and reconfigurations are propagated quickly, without the potential for temporary loops or any delays due to prolonged waiting times for route convergence.

    Indirect contributors to availability are of course the improved security and isolation properties is SCION. Misconfigurations can no longer spread outside an ISD, nor can attacks. The split of data and control plain further improves availability by making sure that the dataplane may continue to function, even when the control plaine is disrupted.
    
    \subsection{Trust Management}
    An effort is made to move the trust model in SCION towards one which is more meaningful to humans, than the current ones at work in TLS PKI, DNSsec or BPGSec. This achived first and for most by drastically reducing the cyrcle of trust by reducing the number of trust roots a user needs to rely upon through the introduction of ISDs.. This in turn enables trust agility by making quick and effectiv key revocation feasable. Further more, ISDs are engineered to model and conform to existing trust boundaries derived from political and comerial realworld structures. At the sametime the single point of failure problem is avoided by introducing accountability into the management of trust rootsthrough the implementation of ARPKI \cite{rpki}.

    \subsection{Data Integrity and Accountability}
    While SCION leaves data integrity and data privacy of the actual payload data to other layers, host of cryptographic measure are employed to ensure the data it produces and consumes can not be tampered with. PCBs are signed each time they are forwarded, so a malicious AS can not alter any part of of a received PCB, nor can it advertise links it has no rights to with out detection. A misbehaving AS can always be attributed by its signature and can either circumvented by source routing or its keys revoked by the ISD core.

    Since routing information is contained in each packets header, this needs to be protected as well. SCION protects each entry in the opaque field of each path-segment with a MAC, produced by a per AS key. This means a malicious as can not alter the opaque fields generated by other ASes.

    By implementing ARPKI \cite{arpki} the certificate authority and by extension the TRC become tamperprove as well, as an attacker always needs to compromise mayority core ASes in order to approve actions to alter the TRC.

    \subsection{Scalability and Efficiency}
    As propesed in the introduction scalability depends on how easy it is to add and remove entities from a structure and how far these changes must propagate within the structure. SCION manages to keep changes simple and local by introducing ISDs. For a hypothetical world wide depolyment it is expeted that there will eventually be around 6 to 10 \cite{scion_2011} top-level ISDs, each subdevided into smaller subISDs. Consequenlty adding and removing ISDs on any level is strictly contained to the new ISDs peers and its parent.

    Adding and removing ASes from an ISD is contained in a similar manner. As described in \ref{ssec:isd_components} an AS joins an ISD by purchasing connectivety from an AS which is allready member of the ISD the AS wishes to joint. To only entities involved are one or multiple provider ASes and the ISD core which needs to register the new paths in its path servers and signe the new ISD keys.

    A hughe boon for efficiency is the fact, that SCION removes the neccessity of routing tables in border routers, by storing forwarding information in the packet header. Current BGP routers use specialiced memory architectures to store the ever larger routing tables and perform prefix matching an acceptable speed. This special hardware is not only expensive but also power hungry. When comparing BGP und SCION routers in a simuliton XY et al. postulate an overal power saving of at least 16 \%. They also find that the impact larger packet headers due to the PCFS inforamtion is negliable \cite{scion_power}.


\end{document}
