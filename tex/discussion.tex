\documentclass[../eva1_scion.tex]{subfiles}
\begin{document}
   \section{Discussion}\label{sec:discussion}

    After we have looked at the components and processes which constitute the SCION architecture, in this section we discuss how SCION manages to provide the quality metrics defined in the introduction (see section \ref{sec:introduction}) and look at further emergent properties of the architecture.

    \subsection{Availability}
    SCION incorporates multiple direct design decisions which work towards improving availability of the network, as well as other properties of the design which indirectly contribute towards that goal.

    In SCION's down-path registration mechanism makes SCION inherently multipathed, since each AS may register multiple down-paths with the core path servers. This control over down paths also allows an AS to select down paths for their reliability, and may select paths in a way to avoid unreliable ASes. Further,  the end host may take reliability into account during path construction as well. Finally, each packet carries its routing information in the packet header and the source may select a path on a per-packet basis, making use of the multipathed nature of SCION.

    The high path freshness by guaranteed by SCION's regular beaconing process further works in favour of availability, since valid and working paths are propagated through an ISD every few seconds. Physical route failures are detected quickly and reconfigurations are propagated quickly, without the potential for temporary loops or any delays due to prolonged waiting times for route convergence.

    Indirect contributors to availability are of course the improved security and isolation properties is SCION. Misconfigurations can no longer spread outside an ISD, nor can attacks. The split of data and control plain further improves availability by making sure that the data plane may continue to function, even when the control plane is disrupted.

    \subsection{Trust Management}
    An effort is made to move the trust model in SCION towards one which is more meaningful to humans, than the current ones at work in TLS PKI, DNSsec or BPGSec. This achieved first and for most by drastically reducing the cycle of trust by reducing the number of trust roots a user needs to rely upon through the introduction of ISDs. This in turn enables trust agility by making quick and effective key revocation feasible. Furthermore, ISDs are engineered to model and conform to existing trust boundaries derived from political and commercial real-world structures. At the same time, the single point of failure problem is avoided by introducing accountability into the management of trust roots through the implementation of ARPKI \cite{arpki_2014, arpki_2018}.

    \subsection{Data Integrity and Accountability}
    While SCION leaves data integrity and data privacy of the actual payload data to other layers, host of cryptographic measure are employed to ensure the data it produces and consumes can not be tampered with. PCBs are signed each time they are forwarded, so a malicious AS can not alter any part of  a received PCB, nor can it advertise links it has no rights to without detection. A misbehaving AS can always be attributed by its signature and can either circumvent by source routing or its keys revoked by the ISD core.

    Since routing information is contained in each packet header, this needs to be protected as well. SCION protects each entry in the opaque field of each path-segment with a MAC, produced by a per AS key. This means a malicious as cannot alter the opaque fields generated by other ASes.

    By implementing ARPKI the certificate authority and by extension the TRC become tamper prove as well, as an attacker always needs to compromise majority core ASes to be able to approve actions which alter the TRC.

    \subsection{Scalability and Efficiency}
    As proposed in the introduction scalability depends on how easy it is to add and remove entities from a structure and how far these changes must propagate within the structure. SCION manages to keep changes simple and local by introducing ISDs. For a hypothetical worldwide deployment, it is expected that there will eventually be around 6 to 10 \cite{scion_2011} top-level ISDs, each subdivided into smaller subISDs. Consequently, adding and removing ISDs on any level is strictly contained to the new ISDs peers and its parent.

    Adding and removing ASes from an ISD is contained similarly. As described in \ref{sssec:as_componants} an AS joins an ISD by purchasing connectivity from an AS which is already a member of the ISD the AS wishes to join. So, the only entities involved are one or multiple provider ASes and the ISD core, which needs to register the new paths in its path servers and sign the new ISD's keys.

    A massive boon for efficiency is the fact, that SCION removes the necessity of routing tables in border routers, by storing forwarding information in the packet header. Current BGP routers use specialized memory architectures to store the ever larger routing tables and perform prefix matching at acceptable speed. This special hardware is not only expensive but also power hungry. When comparing BGP and SCION routers in a simulation, Chen et al. postulate an overall power saving of at least 16 \%. They also find that the impact of larger packet headers due to the PCFS information is neg liable \cite{scion_power}.

\end{document}
