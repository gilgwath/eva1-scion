\documentclass[../eva1_scion.tex]{subfiles}
\begin{document}
    \section{Related Work}\label{sec:related_work}

    Xin Hhang et al. lay the groundwork in specifying SCION in their 2011 paper titled "SCION: Scalability, Control, and Isolation On Next-Generation Networks
" \cite{scion_2011}. This paper is later updated and referenced in a follow-up paper in 2015 where Barrera et al. who revisit SCION \cite{scion_2015}, while in their 2017 paper they give a detailed and updated overview in the Communication of the ACM \cite{scion_2017}. Multiple extension to the SCION protocol were proposed as follows. Onion routing was introduced through an extension described by Chen et al. \cite{hornet_2015, hornet_2016} and effective defences against DDoS attacks are discussed by Basescu et al. \cite{sibra_2016}, Lee et al. \cite{lee_2017} and Rothenberger et al. \cite{piskes_2020}.

Since SCION heavily relies on multiple PKI systems, therefore Basin et. all propose \cite{arpki_2014} and implement \cite{arpki_2018} an attack resilient PKI model. In order to simplify and better secure the PKI duties which come with operating a SCION IDS Matsumote et. al. introduce "CASTLE: CA Signing in a Touch-Less Environment" \cite{castle_2016}.

Ding et al. analyse five next-generation networking protocols including SCION and compare them in 2016 study presented during  IEEE Access \cite{ding_2016} and Know et al. present their findings from a real-world test bed running since 2016 ICNP 2020 \cite{testbed_2020}. A study by Giacomo et al. even suggests potential applications for SCION-based networks in currently emerging internet satellite constellations \cite{giuliari_2020}.

\end{document}
