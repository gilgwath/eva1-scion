\documentclass[../eva1_scion.tex]{subfiles}
\begin{document}
\section{Introduction}\label{sec:introduction}
\setcounter{page}{1}
\pagenumbering{arabic}

Todays world is extremely connected and almost every branch of live heavily depends on the constant availability and security of data connections. What the social and economic inpact of an just a brief and local outage is, is hard to tell and becomes even harder to grasp if one considers a global outage going on for hourse or days at a time. As individuals we seldomly experience outages, which may lead one to the assumption that the internet is a reliable and highly available construct. However, upon closer inspection we are forced to the conclusion that it is neither enigneered with that goal in mind, nor does it provide this quality an practice. Further more, at least since the Snowden revelations, we also know that security, privacy and trust are in even more fragile state than the internets availability.

These shortcommings are deeply rooted in the arichtecture of the internet, since it has now grown far beyond the wildest assumptions at time of its insception in the 1970s. When the internet and its core protocols were designed nobody cloud imagine the eventual scale it would reach, also solving the technical challenge of connecting computers over long distances reliably took presedence over matters like security and efficiency. In fact getting it to work at all was seen as a major achievement. By the early 1990s the internet as we know it today has come together. Since then it has evolved little and if so reluctantly. As a consequence todays protocols are no longer up to the task of managiging the scale and complexity of the internet, nor navigating the modern threat landscape.

In this paper we give an introduction to \textit{scalabilitiy, control, and isolation on next-generation networks} also known as \textit{ SCION }. A project which aims to provide a clean slate reengineering of the core interntet infrastructure, in order to solve some of the most pressing concerns which plague the modern global internet. This paper will outline inherent shortcommings of the current, mostly BPG based, internet architecture and examine how SCION proposes to solve said challanges. We will explore the concepts contained in the proposed solution as well, as comparing SCION to the BGP in multiple aspects. Our findigs will be present in a easy to read form for fellow professionals familiar with the fundamentals of networking, but unfamiliar with SCION, thus giving an introduction into the subjectmatter.

\end{document}
