\documentclass[../eva1_scion.tex]{subfiles}
\begin{document}
\section{Introduction}\label{sec:introduction}

    Today's world is extremely connected, and almost every branch of live heavily depends on the constant availability and security of data connections. What the social and economic impact of just a brief and local outage are going to do is hard to tell, and becomes even harder to grasp if one considers a global outage going on for hours or days at a time. As individuals, we seldom experience outages or attacks, which may lead one to the assumption that the internet is a reliable and highly available construct, which offers private and secure communications. However, upon closer inspection we are forced to the conclusion that it is neither engineered with the current scale in mind, nor does it provide these qualities reliably in practice. Furthermore, at least since the Snowden revelations, we also know that security, privacy, and trust are in an even more fragile state than the internet's availability.

    These shortcomings are deeply rooted in the architecture of the internet, since it has now grown far beyond the wildest assumptions at the time of its inception in the 1970s. When the internet and its core protocols were designed nobody cloud imagine the eventual scale it would reach, actually solving the technical challenge of connecting computers over long distances reliably took precedence over matters like security, privacy and efficiency. In fact, making it work at all was seen as a major achievement. By the early 1990s, the internet as we know it today has come together. Since then, it has evolved little, and if so reluctantly. As a consequence, the current protocols are no longer up to the task of managing the scale and complexity of the internet, nor navigating the modern threat landscape.

    In this paper, we give an introduction to \textit{scalabilitiy, control, and isolation on next-generation networks} also known as SCION. The project  aims to provide a clean slate reengineering of the core internet infrastructure, in order to solve some of the most pressing concerns which plague the modern internet. This paper will outline the inherent shortcomings of the current, mostly BPG-based internet architecture and examine how SCION proposes to solve said challenges. We will explore the concepts contained in the suggested solution, as well as comparing SCION to  BGP in multiple aspects. Our findings will be presented in an easy-to-read form for fellow professionals familiar with the fundamentals of networking, but unfamiliar with SCION, thus giving an introduction into the subject.

\end{document}
