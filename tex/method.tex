\documentclass[../eva1_scion.tex]{subfiles}
\begin{document}
    \chapter{Method}\label{ch:method}

   This paper is purely based an survey of existing literature. It's main purpose is to digest existing literature into a form easily accessible to other IT professionals familiar with the basics of networking but unfamiliar with SCION.

   Literature for this paper was mainly selected from the publications section of the SCION project website \cite{scion_website}. We followed an itterative drill down approach, in each itteration skimming a handfull of papers and selecting some for indepth study. After skimming the 32 listed publications on the project website we selected a group of three overview papers covering insception and evolution SCION for indepth study \cite{scion_2011, scion_2015, scion_2017}. During this indepth study we collected references to related and cited works for later review. Having gained a good overview of the topic we proceeded to repeat this process twice, applying the following the selection criteria:

   \begin{enumerate}
       \item The paper clearifies or enhances excisting feature of the SCION architecture.
       \item The paper specifies an extension to the SCION protocol to add optional features or properties.
       \item The paper treats the implementation on an conceptional level of the SCION architecture.
       \item The paper treats a techonolgy or mechanism addopted in SCION.
       \item The paper is citet by the authors an important precursor or competitor to SCION.
       \item The paper does not treat an implemetation of SCION on the code level.
       \item The paper does not reference SCION as an implementation detail towards some other goal.
   \end{enumerate}

   Additionally Google Schoolar \cite{google_scholar}, the IEEE explorer \cite{i3e_explorer} and Science Direct \cite{science_direct} were all searched by the keywords "SCION internet architecture". All three searche engines combined turned up around 3000 results which needed to be dedouplicated and graded as for their relevance according to the above criteria. Unfortunatelty, but somewhate expectedly, this search did turn up some intressitng realted work outside the scope of this paper, but did not bring to light any further literature to be included in our introductory paper.

\end{document}
