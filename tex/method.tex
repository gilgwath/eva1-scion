\documentclass[../eva1_scion.tex]{subfiles}
\begin{document}

\section{Method}\label{sec:method}

    This paper is based a survey of existing literature. Its main purpose is to digest existing literature into a form easily accessible to other IT professionals familiar with the basics of networking but unfamiliar with SCION.

       Literature for this paper was mainly selected from the publications section of the SCION project website \cite{scion_website}. We followed an iterative drill-down approach, in each iteration skimming a handful of papers and selecting some for in-depth study. After skimming the 32 listed publications on the project website, we selected a group of three overview papers covering inception and evolution SCION for in-depth study \cite{scion_2011, scion_2015, scion_2017}. During this in-depth study, we collected references to related and cited works for later review. Having gained a good overview of the topic, we proceeded to repeat this process twice, applying the following the selection criteria:

    \begin{enumerate}
        \item The paper clarifies or enhances existing feature of the SCION architecture.
        \item The paper specifies an extension to the SCION protocol to add optional features or properties.
        \item The paper treats the implementation on a conceptional level of the SCION architecture.
        \item The paper treats a technology or mechanism adopted in SCION.
        \item The paper is cited by the authors an important precursor or competitor to SCION.
        \item The paper does not treat an implementation of SCION on the code level.
        \item The paper does not reference SCION as an implementation detail towards some other goal.
    \end{enumerate}

    Additionally, Google Scholar \cite{google_scholar}, the IEEE explorer \cite{i3e_explorer} and Science Direct \cite{science_direct} were all searched by the keywords "SCION internet architecture". All three searcher engines combined turned up around 3000 results which needed to be deduplicated and graded as for their relevance according to the above criteria. Unfortunately, but somewhat expectedly, this search did turn up some interesting related work outside the scope of this paper, but did not bring to light any further literature to be included in our introductory paper.

\end{document}
